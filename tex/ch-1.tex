\section{ЗАГАЛЬНІ ПОЛОЖЕННЯ}
\subsection{Опис предметного середовища}
\subsubsection{Опис процесу діяльності}

Робота у системі починається із запуску виконуваного файлу, в результаті чого користувач потрапляє до графічного інтерфейсу (рисунок \ref{fig:ui}) у вигляді невеличкого вікна із двома вкладками. У цьому вікні користувач може виконувати конфігурування, управління профілями, переглядати історію, та запускати тренування.

\begin{figure}[here]
  \centering\includegraphics[scale=0.6]{./diagrams/memspacer-ui.png}
  \caption{Інтерфейс користувача}
  \label{fig:ui}
\end{figure}

За замовченням обрано деякий стандартний профіль, та користувач може створювати нові і змінювати поточний за допомогою меню у верхній частині інтерфейсу. До профілів прив'язані налаштування та історія тренувань.

На вкладці ``Налаштування'' наявні елементи управління, за допомогою яких користувач може змінювати параметри системи: як косметичні, такі як кольори та присутність фонового шейдеру, так і предметні, такі як довжина буферів N, використовані ознаки, дозволені просторові перетворення, тощо.

На вкладці ``Історія тренувань'' наявна таблиця із записами, що описують збережені у базі дані про попередні сеанси тренувань для поточного профілю, такі як кількість правильних позитивних відповідей, кількість пропущених позитивних відповідей, кількість неправильних відповідей, тощо.

При натисненні на кнопку ``Запуск'' запускається у режимі повного екрану програма моделювання вправи (рисунок \ref{fig:memspacer}) і починається тренування, яке полягає у тому, що користувачу послідовно показується ряд груп образів, і на кожному кроці користувач повинен визначити, які з поточно-показаних образів були показані рівно N кроків назад, натиснувши відповідні кожному з них кнопки.

\begin{figure}[here]
  \centering\includegraphics[scale=0.25]{./diagrams/memspacer.png}
  \caption{Програма моделювання вправи}
  \label{fig:memspacer}
\end{figure}

Після завершення сеансу тренування дані заносяться у базу в контексті поточного профілю, а користувач потрапляє назад до графічного інтерфейсу, з оновленою вкладкою історії тренувань. Робота із програмою завершується по натисненню на кнопку ``Вихід'' у інтерфейсі.

На рисунку \ref{fig:activity} процес діяльності представлено у вигляді структурної схеми.

\begin{figure}[here]
  \centering\includegraphics[scale=0.5]{./diagrams/activity.eps}
  \caption{Схема структурна діяльності}
  \label{fig:activity}
\end{figure}

\subsubsection{Опис функціональної моделі}

У даній системі єдиним актором є користувач -- людина, що конфігурує та використовує систему задля тренувань. Користувач може виконувати наступні дії:

\begin{enumerate}
  \item конфігурування:
    \begin{enumerate}[leftmargin=24pt]
      \item зміна режиму:
        \begin{itemize}[leftmargin=24pt]
          \item вибір режиму ``Звичайний K-N-Back'';
          \item вибір режиму ``K-N-Back з обертами і здвигами за модулем'';
        \end{itemize}
      \item зміна довжини буферів ознак (N);
      \item вибір додаткових просторових перетворень:
        \begin{itemize}[leftmargin=24pt]
          \item (де)активація Z-обертів лицевою гранню;
        \end{itemize}
      \item вибір додаткових ознак:
        \begin{itemize}[leftmargin=24pt]
          \item (де)активація використання ознаки кольору блимання клітин;
          \item (де)активація використання ознаки звуку оголошеної літери;
        \end{itemize}
      \item зміна графічних опцій:
        \begin{itemize}[leftmargin=24pt]
          \item де(активація) використання космічного фону;
          \item зміна кольору активації (клітин);
          \item зміна кольору спокою (клітин);
        \end{itemize}
      \item зміна таймінгів:
        \begin{itemize}[leftmargin=24pt]
          \item зміна часу активації;
          \item зміна інтервалів;
        \end{itemize}
    \end{enumerate}
  \item управління профілями:
    \begin{enumerate}[leftmargin=24pt]
      \item вибір поточного профілю;
      \item створення профілю;
      \item видалення профілю;
      \item збереження змін профілю;
      \item відміна змін профілю;
    \end{enumerate}
  \item перегляд історії тренувань;
  \item тренування:
    \begin{enumerate}[leftmargin=24pt]
      \item запам'ятання ознак:
        \begin{itemize}
          \item запам'ятання позиції;
          \item запам'ятання кольору;
          \item запам'ятання звуку;
        \end{itemize}
      \item сигналізування співпадіння ознак:
        \begin{itemize}[leftmargin=24pt]
          \item сигналізування співпадіння позиції;
          \item сигналізування співпадіння кольору;
          \item сигналізування співпадіння звуку.
        \end{itemize}
    \end{enumerate}
\end{enumerate}

Структурну схему варіантів використань наведено у графічному додатку.

\begin{comment}
\begin{figure}[here]
  \centering\includegraphics[scale=0.5]{./diagrams/usecase-top.eps}
  \caption{Діаграма прецедентів верхнього рівня}
  \label{fig:usecase-top}
\end{figure}

\begin{figure}[here]
  \centering\includegraphics[scale=0.6]{./diagrams/usecase-config.eps}
  \caption{Діаграма прецедентів конфігурування}
  \label{fig:usecase-config}
\end{figure}

\begin{figure}[here]
  \centering\includegraphics[scale=0.6]{./diagrams/usecase-profiles.eps}
  \caption{Діаграма прецедентів управління профілями}
  \label{fig:usecase-profiles}
\end{figure}

\begin{figure}[here]
  \centering\includegraphics[scale=0.6]{./diagrams/usecase-training.eps}
  \caption{Діаграма прецедентів тренування}
  \label{fig:usecase-training}
\end{figure}
\end{comment}

\newpage
\subsection{Огляд наявних аналогів}

\subsubsection{Браузерні реалізації}

У таблиці \ref{table:browser-impls} наведено перелік браузерних аналогів.

\small\begin{longtable}{| C{5cm} | C{2cm} | C{2.5cm} | C{5cm} |}
  % \captionsetup{margin=2cm}
  \caption{Браузерні аналоги системи}
  \label{table:browser-impls} \\
  \hline
  Назва/ресурс & Тип & Платформа & Особливості \\
  \hline
  Ntllct N-Back\cite{ntllct}
  & 3-N-Back
  & Браузер
  & У якості ознак використовуються позиція, оголошена літера і колір \\
  \hline
  Soak Your Head\cite{soak-your-head}
  & 2-N-Back
  & Браузер
  & У якості ознак використовуються позиція і оголошена літера \\
  \hline
  Cognitive Fun\cite{cognitive-fun}
  & 2-1-Back, 2-2-Back
  & Браузер
  & У якості ознак використовуються позиція і оголошена літера \\
  \hline
  Brainexer\cite{brainexer}
  & K-N-Back з K від 2 до 5
  & Браузер
  & Усі ознаки візуальні і відображаються набором із K самостійних зображень.
  Користувач вводить не співпадання кожної окремої ознаки, а цифру – їх кількість.
  Ознаки: фігура, колір, позиція, літера, цифра \\
  \hline
\end{longtable}\normalsize
\newpage
\subsubsection{Мобільні реалізації}

У таблиці \ref{table:mobile-impls} наведено перелік мобільних аналогів.

\small\begin{longtable}{| C{5cm} | C{2cm} | C{2.5cm} | C{5cm} |}
  \caption{Мобільні аналоги системи}
  \label{table:mobile-impls} \\
  \hline
  Назва & Тип & Платформа & Особливості \\
  \hline
  Efrac\cite{efrac}
  & 2-N-Back
  & Android, Blackberry, Nokia
  & У якості ознак використовуються позиція і оголошена літера. \\
  \hline
  N-Back Suite\cite{n-back-suite}
  & 1-N-Back, 2-N-Back
  & Iphone
  & Реалізація з конфігуруємою швидкістю та вибором ознак:
  позиція, зображення, колір, літера (писана чи оголошена) \\
  \hline
  Phuc Dual-N-Back\cite{phuc-dual-n-back}
  & K-N-Back з K від 2 до 4
  & Android
  & Ознаки: позиція, звук (на вибір літери, чи піаніно), колір, зображення \\
  \hline
\end{longtable}\normalsize
\newpage
\subsubsection{Реалізації на PC}

У таблиці \ref{table:pc-impls} наведено перелік комп'ютерних аналогів.

\small\begin{longtable}{| C{3.5cm} | C{4cm} | C{1.8cm} | C{5.2cm} |}
  \caption{Комп'ютерні аналоги системи}
  \label{table:pc-impls} \\
  \hline
  Назва/ресурс & Тип & Платформа & Особливості \\
  \hline
  Brain Workshop\cite{brainworkshop}
  & K-N-Back з K від 1 до 6; Арифметичний N-Back; Варіабельний N-Back;
  Крабовий N-Back; Мультиобразний N-Back
  & Windows, Linux, MacOS
  & Дуже конфігуруєма і гнучка реалізація. Окрім різноманітних K-N-Back з K від 1 до 6
  з такими ознаками, як позиція, колір, символ, звук
  (на вибір: літери, цифри, фонетичний алфавіт NASA, піаніно, коди Морзе),
  присутні також нестандарні модифікації \\
  \hline
  Memspacer
  & Spatial K-N-Back
  & Windows, Linux
  & У даному варіанті дошка є трьовимірною і має структуру кубіка Рубіка, тобто великого куба,
  що складається з менших. Кожна грань цього куба є дошкою для K-N-Back,
  а нові мікромодифікації полягають у можливості зіставлення трьовимірним рухам дошки інформації,
  яку користувач має прийняти до уваги \\
  \hline
\end{longtable}\normalsize

Основним недоліком усіх наведених аналогів у порівнянні із системою ``Memspacer'' є двовимірність поля, другорядними недоліками багатьох із них також є маленька кількість доступних ознак і відсутність гнучкої конфігуруємості.

\newpage
\subsection{Постановка задачі}

Комплекс задач призначений для покращення короткочасної пам'яті і корельованих із нею характеристик.

Цілі розробки:
\begin{itemize}
  \item покращення короткочасної пам'яті;
  \item покращення здатності до концентрації уваги;
  \item покращення рухливого інтелекту.
\end{itemize}

Задачі розробки:
\begin{itemize}
  \item моделювання вправи у реальному часі;
  \item відтворення ознак графікою та звуком;
  \item ведення історії тренувань.
\end{itemize}

\textbf{Висновок до розділу.} У даному розділі було детально розглянуто предметне середовище системи: проаналізовані процеси діяльності користувача і функціональна модель системи, побудовані відповідні структурні схеми. Також були розглянуті аналоги і описана їх функціональність. Були сформульовані призначення, цілі, та задачі розробки.