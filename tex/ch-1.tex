\section{Загальні положення}
\subsection{Опис предметного середовища}

Об'єкт моделювання представляє з себе просторову варіацію вправи для розвитку короткочасної пам'яті N-Back, реалізовану у формі інтерактивної гри. Оригінальну вправу винайшов Уейн Кірхнер у 1958 році\cite{kirchner} як задачу неперервного виконання для оцінки короткочасної пам'яті з когнітивно-нейронаукових позицій.

Задача у найбільш загальній формі — назвемо її K-N-Back — полягає у тому, що суб'єкту надається ряд із K послідовностей ознак, і він повинен реагувати на повтори ознак у рамках своїх рядів, тобто, на кожному N-му кроці відповідати на питання, чи зустрічалася дана ознака для даного ряду N кроків назад. Таким чином, у термінології комп'ютерних наук, вправа полягає у тому, щоб тримати у пам'яті K буферів довжини N, на кожному кроці класти у кожний з них по новому елементу, здвигаючи усі інші і відкидуючи останні.

Найбільш популярну варіацію вправи — 2-N-Back — запропонувала Сюзанна Йеггі у 2003 році\cite{jaeggi}, як застосування до вправи парадигми подвійної задачі. Ця парадигма полягає у виконанні двох задач одночасно і подальших порівнянь із продуктивністю на кожній з задач окремо. Якщо результат погіршується, то задачі перетинаються, і вважається, що вони потребують ресурси мозку з одного і того ж класу щодо потреб з обробки інформації.

Більшість із поширених комп'ютерних реалізацій вправи моделюють її 2-N-Back варіант, інші - дозволяють у деяких діпазонах змінювати K та N, деякі підтримують більш екзотичні модифікації, та усі є двовимірними щодо найбільш популярної ознаки — позиції на дошці. У даній роботі дошка є трьовимірною і має структуру кубіка Рубіка, тобто великого куба, що складається з менших. Кожна грань цього куба спіставима із двовимірною дошкою для K-N-Back, і відкривається цілий простір до нових модифікацій, заснованих на можливості зіставлення трьовимірним рухам дошки інформації, яку користувач має прийняти до уваги.

У науковців досі виникають питання щодо конструктної валідності вправи. У той час як задача має сильну довірчу валідність і знайшла широке застосування у якості міри короткочасної пам'яті у кліничній та експериментальній практиці, існує декілька досліджень, демонструючих помірну конвергентну валідність із іншими мірами короткочасної пам'яті\cite{kane-conway}\cite{jaeggi-buschkuehl}.

Щодо такої кореляції існують дві основних гіпотези: одна полягає у тому, що вправа активізує відмінні від інших субкомпоненти короткочасної пам'яті; інша, більш критична, стверджує, що успішність виконання вправи більше залежить від звички та когнітивних процесів впізнавання. Як би там не було, науковці підкреслюють необхідність подальшого дослідження конструктної валідності N-Back.

Нейробіологічні дослідження показали, що під час виконання вправи N-Back найбільш активними є такі зони мозку: бічна премоторна кора (lateral premotor cortex), дорсальна поясна кора (dorsal cingulate cortex),  середня премоторна кора (medial premotor cortex), дорсолатеральна та вентролатеральна префронтальна кора (dorsolateral and ventrolateral prefrontal cortex), фронтальні полюси (frontal poles), медіальна і латеральна задня тім'яна кора (medial and lateral posterior parietal cortex).

\begin{comment}
У багатьох реалізаціях N-Back (наприклад, Brain Workshop) наявна конфігуруємість, у якій доступні ті чи інші модифікації класичної вправи. Якщо їх назвати мікромодифікаціями, то основну модифікацію даного продукту — просторовість — по відношенню до них можна буде назвати макромодифікацію у тому сенсі, що вона відкриває цілий вимір для нових мікромодифікацій.

Класичну вправу та деякий клас її мікромодифікацій можна узагальнити до K-N-Back, сутність якого полягає у тому, що користувачу послідовно показується ряд груп образів по K за раз, і на кожному кроці користувач повинен визначити, які з поточно-показаних K образів були показані рівно N кроків назад, натиснувши відповідні кожному з них кнопки.

У найбільш поширеному варіанті вправи — Dual-N-Back — як можна здогадатися, K дорівнює 2. Як правило, образами у реалізації такого варіанту є візуальний — підсвічена клітина на дошці, та аудіо — озвучена літера.

У даному варіанті дошка є трьовимірною і має структуру кубіка Рубіка, тобто великого куба, що складається з менших. Кожна грань цього куба є дошкою для K-N-Back, а нові мікромодифікації полягають у можливості зіставлення трьовимірним рухам дошки інформації, яку користувач має прийняти до уваги.
\end{comment}

Цільовою аудиторією є будь-які люди, які хочуть покращити короткочасну пам'ять за тих чи інших причин. У клінічній практиці класичний варіант вправи використовується для підвищення здатності до концентрації і короткочасного запам'ятовування у людей з синдромом дефіциту уваги і гіперактивності та для реабілітації здібностей у людей з травмами мозку.

\subsection{Опис процесу діяльності}

Робота у системі починається із запуску виконуваного файлу, в результаті чого користувач потрапляє до графічного інтерфейсу у вигляді невеличкого вікна із двома вкладками. У цьому вікні користувач може виконувати конфігурування, управління профілями, переглядати історію, та запускати тренування.

За замовченням обрано деякий стандартний профіль, та користувач може створювати нові і змінювати поточний за допомогою меню у верхній частині інтерфейсу. До профілів прив'язані налаштування та історія тренувань.

На вкладці ``Налаштування'' наявні елементи управління, за допомогою яких користувач може змінювати параметри системи: як косметичні, такі як кольори та присутність фонового шейдеру, так і предметні, такі як довжина буферів N, використовані ознаки, дозволені просторові перетворення, тощо.

На вкладці ``Історія тренувань'' наявна таблиця із записами, що описують збережені у базі дані про попередні сеанси тренувань для поточного профілю, такі як кількість правильних позитивних відповідей, кількість пропущених позитивних відповідей, кількість неправильних відповідей, тощо.

При натисненні на кнопку ``Запуск'' запускається у режимі повного екрану програма моделювання вправи і починається тренування, яке полягає у тому, що користувачу послідовно показується ряд груп образів, і на кожному кроці користувач повинен визначити, які з поточно-показаних образів були показані рівно N кроків назад, натиснувши відповідні кожному з них кнопки.

Після завершення сеансу тренування дані заносяться у базу в контексті поточного профілю, а користувач потрапляє назад до графічного інтерфейсу, з оновленою вкладкою історії тренувань. Робота із програмою завершується по натисненню на кнопку ``Вихід'' у інтерфейсі.

\begin{figure}[here]
  \caption{Діаграма діяльності}
  \centering\includegraphics[scale=0.6]{./diagrams/activity.eps}
  \label{fig:activity}
\end{figure}

\subsection{Опис функціональної моделі}

\subsubsection{Актори}
У даній системі єдиним актором є користувач -- людина, що конфігурує та використовує систему задля тренувань.

\subsubsection{Прецеденти}

\begin{itemize}
  \item Управління профілями:
    \begin{itemize}[leftmargin=24pt]
      \item Вибір поточного профілю
      \item Створення профілю
      \item Видалення профілю
      \item Збереження змін профілю
      \item Відміна змін профілю
    \end{itemize}
  \item Конфігурування:
    \begin{itemize}[leftmargin=24pt]
      \item Зміна режиму:
        \begin{itemize}[leftmargin=24pt]
          \item Вибір режиму ``Звичайний K-N-Back''
          \item Вибір режиму ``K-N-Back з обертами і здвигами за модулем''
        \end{itemize}
      \item Зміна довжини буферів ознак (N)
      \item Вибір додаткових просторових перетворень:
        \begin{itemize}[leftmargin=24pt]
          \item (Де)активація Z-обертів лицевою гранню
        \end{itemize}
      \item Вибір додаткових ознак:
        \begin{itemize}[leftmargin=24pt]
          \item (Де)активація використання ознаки кольору блимання клітин
          \item (Де)активація використання ознаки звуку оголошеної літери
        \end{itemize}
      \item Зміна графічних опцій:
        \begin{itemize}[leftmargin=24pt]
          \item Де(активація) використання космічного фону
          \item Зміна кольору активації (клітин)
          \item Зміна кольору спокою (клітин)
        \end{itemize}
      \item Зміна таймінгів:
        \begin{itemize}[leftmargin=24pt]
          \item Зміна часу активації
          \item Зміна інтервалів
        \end{itemize}
    \end{itemize}
  \item Перегляд історії тренувань
  \item Тренування
    \begin{itemize}[leftmargin=24pt]
      \item Запам'ятання ознак
        \begin{itemize}
          \item Запамятання позиції
          \item Запамятання кольору
          \item Запамятання звуку
        \end{itemize}
      \item Сигналізація співпадіння ознак
        \begin{itemize}[leftmargin=24pt]
          \item Сигналізація співпадіння позиції
          \item Сигналізація співпадіння кольору
          \item Сигналізація співпадіння звуку
        \end{itemize}
    \end{itemize}
\end{itemize}

\begin{figure}[here]
  \centering\includegraphics[scale=0.5]{./diagrams/usecase-top.eps}
  \caption{Діаграма прецедентів верхнього рівня}
\end{figure}

\begin{figure}[here]
  \centering\includegraphics[scale=0.6]{./diagrams/usecase-config.eps}
  \caption{Діаграма прецедентів конфігурування}
\end{figure}

\begin{figure}[here]
  \centering\includegraphics[scale=0.6]{./diagrams/usecase-profiles.eps}
  \caption{Діаграма прецедентів управління профілями}
\end{figure}

\begin{figure}[here]
  \centering\includegraphics[scale=0.6]{./diagrams/usecase-training.eps}
  \caption{Діаграма прецедентів тренування}
\end{figure}

\subsection{Огляд наявних аналогів}

\subsubsection{Браузерні реалізації}
\small\begin{longtable}{| C{5cm} | C{2cm} | C{2.5cm} | C{5cm} |}
  \captionsetup{justification=centering}\caption{} \\
  \hline
  Назва/ресурс & Тип & Платформа & Особливості \\
  \hline
  \url{http://www.ntllct.com/index.php?show=games\&gid=2}
  & 3-N-Back
  & Браузер
  & У якості ознак використовуються позиція, оголошена літера і колір \\
  \hline
  \url{http://www.soakyourhead.com/}
  & 2-N-Back
  & Браузер
  & У якості ознак використовуються позиція і оголошена літера \\
  \hline
  \url{http://cognitivefun.net/test/5}
  & 2-1-Back, 2-2-Back
  & Браузер
  & У якості ознак використовуються позиція і оголошена літера \\
  \hline
  \url{http://ru.brainexer.com/nback.html}
  & K-N-Back з K від 2 до 5
  & Браузер
  & Усі ознаки візуальні і відображаються набором із K самостійних зображень.
  Користувач вводить не співпадання кожної окремої ознаки, а цифру – їх кількість.
  Ознаки: фігура, колір, позиція, літера, цифра \\
  \hline
\end{longtable}\normalsize
~\newline
\subsubsection{Мобільні реалізації}
\small\begin{longtable}{| C{5cm} | C{2cm} | C{2.5cm} | C{5cm} |}
  \captionsetup{justification=centering}\caption{} \\
  \hline
  Назва/ресурс & Тип & Платформа & Особливості \\
  \hline
  \url{http://www.efrac.com/iq/}
  & 2-N-Back
  & Android, Blackberry, Nokia
  & У якості ознак використовуються позиція і оголошена літера. \\
  \hline
  \url{http://fingerfriendlysoft.com/index.php#nBackSuite}
  & 1-N-Back, 2-N-Back
  & Iphone
  & Реалізація з конфігуруємою швидкістю та вибором ознак:
  позиція, зображення, колір, літера (писана чи оголошена) \\
  \hline
  \url{https://play.google.com/store/apps/details?id=phuc.entertainment.dualnback}
  & K-N-Back з K від 2 до 4
  & Android
  & Ознаки: позиція, звук (на вибір літери, чи піаніно), колір, зображення \\
  \hline
\end{longtable}\normalsize
~\newline
\subsubsection{Реалізації на PC}
\small\begin{longtable}{| C{3.5cm} | C{4cm} | C{1.8cm} | C{5.2cm} |}
  \captionsetup{justification=centering}\caption{} \\
  \hline
  Назва/ресурс & Тип & Платформа & Особливості \\
  \hline
  \url{http://brainworkshop.sourceforge.net/}
  & K-N-Back з K від 1 до 6; Арифметичний N-Back; Варіабельний N-Back;
  Крабовий N-Back; Мультиобразний N-Back
  & Windows, Linux, MacOS
  & Дуже конфігуруєма і гнучка реалізація. Окрім різноманітних K-N-Back з K від 1 до 6
  з такими ознаками, як позиція, колір, символ, звук
  (на вибір: літери, цифри, фонетичний алфавіт NASA, піаніно, коди Морзе),
  присутні також нестандарні модифікації \\
  \hline
  Memspacer
  & Spatial K-N-Back
  & Windows, Linux
  & У даному варіанті дошка є трьовимірною і має структуру кубіка Рубіка, тобто великого куба,
  що складається з менших. Кожна грань цього куба є дошкою для K-N-Back,
  а нові мікромодифікації полягають у можливості зіставлення трьовимірним рухам дошки інформації,
  яку користувач має прийняти до уваги \\
  \hline
\end{longtable}\normalsize

\newpage
\subsection{Постановка задачі}

\subsubsection{Призначення розробки}

Тренувальна система “Memspacer” призначена для:

\begin{itemize}
  \item Тренування короткочасної пам'яті
  \item Тренування здатності до концентрації уваги
  \item Тренування рухливого інтелекту
  \item Збереження і відображення історії тренувань
\end{itemize}

\subsubsection{Цілі розробки}

\begin{itemize}
  \item Впровадження варіанту вправи N-Back із трьовимірною дошкою
  \item Впровадження просторових модифікацій вправи за допомогою трьовимірної дошки
  \item Досягнення зручності, ефективності, компактності, і кросплатформовості системи
\end{itemize}

\subsubsection{Задачі розробки}

\begin{itemize}
  \item Створення інтерфейсу користувачу
  \item Створення бази даних для історії тренувань
  \item Створення спеціалізованого движку для ефективної реалізації вправи
  \item Реалізація варіанту вправи на створеному движку
  \item Інтеграція користувацького інтерфейсу, бази даних, та програми моделювання
\end{itemize}