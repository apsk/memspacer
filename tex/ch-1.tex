\section{Загальні положення}
\subsection{Опис предметного середовища}

Об'єкт моделювання представляє з себе просторову варіацію вправи для розвитку короткочасної пам'яті N-Back, реалізовану у формі інтерактивної гри. Оригінальну вправу винайшов Уейн Кірхнер у 1958 році\cite{kirchner} як задачу неперервного виконання для оцінки короткочасної пам'яті з когнітивно-нейронаукових позицій.

Задача у найбільш загальній формі — назвемо її K-N-Back — полягає у тому, що суб'єкту надається ряд із K послідовностей ознак, і він повинен реагувати на повтори ознак у рамках своїх рядів, тобто, на кожному N-му кроці відповідати на питання, чи зустрічалася дана ознака для даного ряду N кроків назад. Таким чином, у термінології комп'ютерних наук, вправа полягає у тому, щоб тримати у пам'яті K циклічних буфери довжини N, на кожному кроці класти у кожний з них по новому елементу, здвигаючи усі інші і відкидуючи останні.

Найбільш популярну варіацію вправи — 2-N-Back — запропонувала Сюзанна Йеггі у 2003 році\cite{jaeggi}, як застосування до вправи парадигми подвійної задачі. Ця парадигма полягає у виконанні двох задач одночасно і подальших порівнянь із продуктивністю на кожній з задач окремо. Якщо результат погіршується, то задачі перетинаються, і вважається, що вони потребують ресурси мозку з одного і того ж класу щодо потреб з обробки інформації.

Більшість із поширених комп'ютерних реалізацій вправи моделюють її 2-N-Back варіант, інші - дозволяють у деяких діпазонах змінювати K та N, деякі підтримують більш екзотичні модифікації, та усі є двовимірними щодо найбільш популярної ознаки — позиції на дошці. У даній роботі дошка є трьовимірною і має структуру кубіка Рубіка, тобто великого куба, що складається з менших. Кожна грань цього куба спіставима із двовимірною дошкою для K-N-Back, і відкривається цілий простір до нових модифікацій, заснованих на можливості зіставлення трьовимірним рухам дошки інформації, яку користувач має прийняти до уваги.

У науковців досі виникають питання щодо конструктної валідності вправи. У той час як задача має сильну довірчу валідність і знайшла широке застосування у якості міри короткочасної пам'яті у кліничній та експериментальній практиці, існує декілька досліджень, демонструючих помірну конвергентну валідність із іншими мірами короткочасної пам'яті\cite{kane-conway}\cite{jaeggi-buschkuehl}.

Щодо такої кореляції існують дві основних гіпотези: одна полягає у тому, що вправа активізує відмінні від інших субкомпоненти короткочасної пам'яті; інша, більш критична, стверджує, що успішність виконання вправи більше залежить від звички та когнітивних процесів впізнавання. Як би там не було, науковці підкреслюють необхідність подальшого дослідження конструктної валідності N-Back.

Нейробіологічні дослідження показали, що під час виконання вправи N-Back найбільш активними є такі зони мозку: бічна премоторна кора (lateral premotor cortex), дорсальна поясна кора (dorsal cingulate cortex),  середня премоторна кора (medial premotor cortex), дорсолатеральна та вентролатеральна префронтальна кора (dorsolateral and ventrolateral prefrontal cortex), фронтальні полюси (frontal poles), медіальна і латеральна задня тім'яна кора (medial and lateral posterior parietal cortex).

\begin{comment}
У багатьох реалізаціях N-Back (наприклад, Brain Workshop) наявна конфігуруємість, у якій доступні ті чи інші модифікації класичної вправи. Якщо їх назвати мікромодифікаціями, то основну модифікацію даного продукту — просторовість — по відношенню до них можна буде назвати макромодифікацію у тому сенсі, що вона відкриває цілий вимір для нових мікромодифікацій.

Класичну вправу та деякий клас її мікромодифікацій можна узагальнити до K-N-Back, сутність якого полягає у тому, що користувачу послідовно показується ряд груп образів по K за раз, і на кожному кроці користувач повинен визначити, які з поточно-показаних K образів були показані рівно N кроків назад, натиснувши відповідні кожному з них кнопки.

У найбільш поширеному варіанті вправи — Dual-N-Back — як можна здогадатися, K дорівнює 2. Як правило, образами у реалізації такого варіанту є візуальний — підсвічена клітина на дошці, та аудіо — озвучена літера.

У даному варіанті дошка є трьовимірною і має структуру кубіка Рубіка, тобто великого куба, що складається з менших. Кожна грань цього куба є дошкою для K-N-Back, а нові мікромодифікації полягають у можливості зіставлення трьовимірним рухам дошки інформації, яку користувач має прийняти до уваги.
\end{comment}

Цільовою аудиторією є будь-які люди, які хочуть покращити короткочасну пам'ять за тих чи інших причин. У клінічній практиці класичний варіант вправи використовується для підвищення здатності до концентрації і короткочасного запам'ятовування у людей з синдромом дефіциту уваги і гіперактивності та для реабілітації здібностей у людей з травмами мозку.

\subsubsection{Опис процесу діяльності}

Робота користувача у системі починається з головного меню, де він може вибрати одну з дій:

\begin{enumerate}
  \item Конфігурування
  \item Сеанс тренування
  \item Перегляд історії тренувань
\end{enumerate}

При виборі пункту ``конфігурування'' користувач потрапляє до меню, де він може змінювати параметри системи: як косметичні, такі як кольори та фоновий шейдер, так і предметні, такі як кількість ознак K, довжина буферів N, дозволені просторові перетворення, тощо.

При виборі пункту ``сеанс тренування'' починається тренування, яке полягає у тому, що користувачу послідовно показується ряд груп образів по K за раз, і на кожному кроці користувач повинен визначити, які з поточно-показаних K образів були показані рівно N кроків назад, натиснувши відповідні кожному з них кнопки.

При виборі пункту ``перегляд історії тренувань'' користувач потрапляє до інтерфейсу, що відображає збережені у базі дані про попередні сеанси, такі як кількість правильних відповідей, загальна кількість випробувань, тощо.

\begin{center}
  \textbf{Діаграма діяльності}
  \includegraphics[scale=0.6]{./activity.png}
\end{center}

\subsubsection{Опис функціональної моделі}

\begin{center}
  \textbf{Актори}
\end{center}

\begin{enumerate}
  \item Користувач – людина, що виконує взаємодію з системою.
  \item База даних – сховище, де зберігається історія тренувань.
\end{enumerate}

\begin{center}
  \textbf{Прецеденти}
\end{center}

\begin{enumerate}
  \item Проведення сеансу тренування – це процес утилізації основної функції системи – тренування короткочасної памяті за допомогою просторового варіанту K-N-Back.
  \item Перегляд історії тренувань – перегляд користувачем збережених даних про попередні сеанси, таких як кількість правильних відповідей, загальна кількість випробувань, тощо.
  \item Конфігурування – змінення користувачем параметрів системи: як косметичних, таких як кольори та фоновий шейдер, так і предметних, таких як кількість ознак K, довжина буферів N, дозволені просторові перетворення, тощо.
  \item За допомогою інтерфейсу конфігурування користувач має 	можливість здійснювати наступні впливи на систему:
\end{enumerate}

\begin{itemize}
  \item Вибір альтернативної модифікації вправи;
  \item Зміна N або K;
  \item Вибір задіяних просторових перетворень;
  \item Вибір графічних модифікацій;
\end{itemize}

\begin{center}
  \textbf{Діаграма прецедентів}
  \includegraphics[scale=0.6]{./usecase.png}
\end{center}

\subsection{Огляд наявних аналогів}

Класичну вправу та деякий клас її мікромодифікацій можна узагальнити до K-N-Back, сутність якого полягає у тому, що користувачу послідовно показується ряд груп образів по K за раз, і на кожному кроці користувач повинен визначити, які з поточно-показаних K образів були показані рівно N кроків назад, натиснувши відповідні кожному з них кнопки. Таким чином, у термінології інженерії програмного забезпечення, гра полягає у тому, щоб тримати у пам'яті K буферів довжини N, і на кожному кроці класти у кожний з них по новому елементу, здвигаючи усі інші і відкидуючи останні.

У найбільш поширеному варіанті вправи — Dual-N-Back — як можна здогадатися, K дорівнює 2. Як правило, образами у реалізації такого варіанту є візуальний — підсвічена клітина на дошці, та аудіо — озвучена літера.

\begin{center}
  \vbox{
    \textbf{Браузерні реалізації}
    \begin{tabular}{| p{5cm} | p{2cm} | p{2.5cm} | p{5cm} |}
      \hline
      Назва/ресурс & Тип & Платформа & Особливості \\
      \hline
      \url{http://www.ntllct.com/index.php?show=games\&gid=2}
      & 3-N-Back
      & Браузер
      & У якості ознак використовуються позиція, оголошена літера і колір. \\
      \hline
      \url{http://www.soakyourhead.com/}
      & 2-N-Back
      & Браузер
      & У якості ознак використовуються позиція і оголошена літера. \\
      \hline
      \url{http://cognitivefun.net/test/5}
      & 2-1-Back, 2-2-Back
      & Браузер
      & У якості ознак використовуються позиція і оголошена літера. \\
      \hline
      \url{http://ru.brainexer.com/nback.html}
      & K-N-Back з K від 2 до 5
      & Браузер
      & Усі ознаки візуальні і відображаються набором із K самостійних зображень. Користувач вводить не співпадання кожної окремої ознаки, а цифру – їх кількість. Ознаки: фігура, колір, позиція, літера, цифра. \\
      \hline
    \end{tabular}
  }

  \vbox{
    \textbf{Мобільні реалізації}
    \begin{tabular}{| p{5cm} | p{2cm} | p{2.5cm} | p{5cm} |}
      \hline
      Назва/ресурс & Тип & Платформа & Особливості \\
      \hline
      \url{http://www.efrac.com/iq/}
      & 2-N-Back
      & Android, Blackberry, Nokia
      & У якості ознак використовуються позиція і оголошена літера. \\
      \hline
      \url{http://fingerfriendlysoft.com/index.php#nBackSuite}
      & 1-N-Back, 2-N-Back
      & Iphone
      & Реалізація з конфігуруємою швидкістю та вибором ознак: позиція, зображення, колір, літера (писана чи оголошена). \\
      \hline
      \url{https://play.google.com/store/apps/details?id=phuc.entertainment.dualnback}
      & K-N-Back з K від 2 до 4
      & Android
      & Ознаки: позиція, звук (на вибір літери, чи піаніно), колір, зображення. \\
      \hline
    \end{tabular}
  }

  \vbox{
    \textbf{Реалізації на PC}
    \begin{tabular}{| p{4cm} | p{5cm} | p{2.5cm} | p{3cm} |}
      \hline
      Назва/ресурс & Тип & Платформа & Особливості \\
      \hline
      \url{http://brainworkshop.sourceforge.net/}
      & K-N-Back з K від 1 до 6;  Арифметичний N-Back;  Варіабельний N-Back;  Крабовий N-Back; Мультиобразний N-Back
      & Windows, Linux, MacOS
      & див. далі [1] \\
      \hline
      Memspacer
      & Spatial K-N-Back
      & Windows, Linux
      & див. далі [2] \\
      \hline
    \end{tabular}
  }
\end{center}

[1]: Найбільш конфігуруєма і гнучка реалізація. Окрім різноманітних K-N-Back з K від 1 до 6 з такими ознаками, як позиція, колір, символ, звук (на вибір: літери, цифри, фонетичний алфавіт NASA, піаніно, коди Морзе), присутні також такі нестандарні модифікації, як Арифметичний-N-Back, де необхідно проводити з буфером у пам'яті арифметичні операції, Варіабельний-N-Back, де на кожному кроці N змінюється, Крабовий-N-Back, де кожен блок довжини N треба співставляти з попереднім у зворотньому порядку, та Мультиобразний-N-Back, де треба запам'ятовувати співставлення групам різних візуальних об'єктів на дошці, позиції, у яких кожен конкретний об'єкт з групи виникав.

[2]: У даному варіанті дошка є трьовимірною і має структуру кубіка Рубіка, тобто великого куба, що складається з менших. Кожна грань цього куба є дошкою для K-N-Back, а нові мікромодифікації полягають у можливості зіставлення трьовимірним рухам дошки інформації, яку користувач має прийняти до уваги.


\subsection{Постановка задачі}

\subsubsection{Призначення розробки}

Тренувально-ігрова система “Memspacer” призначена для виконання таких процесів:

\begin{itemize}
  \item Тренування короткочасної памяті;
  \item Тренування здатності до концентрації уваги;
  \item Тренування рухливого інтелекту;
  \item Збереження і відображення історії тренувань;
\end{itemize}

\subsubsection{Цілі та задачі розробки}

\begin{itemize}
  \item Заміщення наявних реалізацій класичного варіанту вправи та її мікромодифікацій, що не надають можливість впровадження до процесу гри рухів у трьовимірному просторі;
  \item Створення реалізації, що є кросплатформенною і підтримує найпоширеніщі операційні системи;
  \item Досягнення ефективності і компактності системи за рахунок використання спеціалізованого під задачу движку, ефективної моделі пам'яті для ігрових об'єктів, та легковагового стеку бібліотек;
\end{itemize}