\section{Математичне забезпечення}

\subsection{Постановка задач}

Під час розробки системи \emph{Memspacer} виникло дві задачі, які потребують нетривіального математичного апарату:
\begin{itemize}
  \item Перехід від координат моделі до координат екрану
  \item Анімація поворотів у просторі із постійною кутовою швидкістю
\end{itemize}

Перша задача вирішена за допомогою матриць перетворень -- зокрема, за допомогою MVP-матриці, а друга -- за допомогою кватерніонів та їх сферичної лінійної інтерполяції.

\subsection{Однорідні координати}

Однорідним називають таке представлення координат, за якого об'єкт, що визначається цими координатами, не змінюється при множенні всіх координат на одне і те ж число. Однорідні координати мають таке ж значення для проективної геометрії як декартові координати для Евклідової геометрії. Поняття однорідних координат було введене Августом Мебіусом у 1827 році у його роботі \emph{``Der barycentrische Calc\"ul''}\cite{mobius},

Однорідні координати вектора $(x,y,z)$ можна записати як четвірку чисел $(x',y',z',w)$, де $x=x'/w$, $y=y'/w$, $z=z'/w$, а $w$ — деяке дійсне число (випадок, коли $w = 0$ є особливим).

Однорідні координати не визначаються однозначно точкою простору. Наприклад, $(1,1,1,1)$ і $(2,2,2,2)$ відповідають одній і той самій точці $(1,1,1)$. При переході до однорідних координат для точки з координатами $(x,y,z)$ беруть точку $(x,y,z,1)$. Зворотіий перехід здійснюється за допомогою ділення на $w$.

\subsection{Геометричні матриці перетворень}

Лінійні перетворення можуть бути виражені у вигляді матриць. Якщо $T$ -- лінійне перетворення, відображаюче $R^n$ у $R^m$, а $\vec x$ -- вектор-стовпець із $n$ записами, тоді
$T( \vec x ) = \mathbf{A} \vec x$
для деякої матриці $\underset{m \times n}{A}$, називається матрицею перетворення $T$.

Матриці перетворень надають можливість описувати довільні лінійні перетворення в одноманітному представленні, що добре підходить для обчислень. Особливо зручним є те, що композиція перетворень у даному форматі відповідає добутку матриць.

Деякі перетворення, що не є лінійними в $n$-мірному просторі, можуть бути виражені як лінійні в $n+1$-мірному просторі. До таких перетворень належать, наприклад, афінні і проективні перетворення. Це є причиною, чому матриці перетворень $4 \times 4$ повсюдно використовуються у трьовимірній комп'ютерній графіці разом із однорідним представленням координат.

\subsubsection{Матриця паралельного перенесення}

Паралельним перенесенням є рух, при якому всі точки пересуваються в одному і тому самому напрямку на одну і ту саму відстань. Цей рух в однорідних координатах може бути описаний матрицею перетворення. Для паралельного перенесення на вектор $\vec v$, однорідний вектор $\vec p$ слід помножити на таку матрицю:

\begin{center}
  $T_{\mathbf{v}} =
  \begin{bmatrix}
    1 & 0 & 0 & v_x \\
    0 & 1 & 0 & v_y \\
    0 & 0 & 1 & v_z \\
    0 & 0 & 0 & 1
  \end{bmatrix}$
\end{center}

Як показано нижче, таке множення дійсно дає очікуваний результат:

\begin{center}
  $T_{\mathbf{v}} \mathbf{p} =
  \begin{bmatrix}
    1 & 0 & 0 & v_x \\
    0 & 1 & 0 & v_y\\
    0 & 0 & 1 & v_z\\
    0 & 0 & 0 & 1
  \end{bmatrix}
  \begin{bmatrix}
    p_x \\ p_y \\ p_z \\ 1
  \end{bmatrix}
  =
  \begin{bmatrix}
    p_x + v_x \\ p_y + v_y \\ p_z + v_z \\ 1
  \end{bmatrix}
  = \mathbf{p} + \mathbf{v}$
\end{center}

\subsubsection{Матриця повороту}

Повороти у трьовимірному просторі також можна описати матрицями перетворень. Нижченаведені матриці здійснюють повороти вектору на кут $\theta$ навколо осей $x$, $y$, та $y$ відповідно:

\begin{center}
  $R_x(\theta) = \begin{bmatrix}
    1 & 0 & 0 \\
    0 & \cos \theta &  -\sin \theta \\[3pt]
    0 & \sin \theta  &  \cos \theta \\[3pt]
  \end{bmatrix}$

  $R_y(\theta) = \begin{bmatrix}
    \cos \theta & 0 & \sin \theta \\[3pt]
    0 & 1 & 0 \\[3pt]
    -\sin \theta & 0 & \cos \theta \\
  \end{bmatrix}$

  $R_z(\theta) = \begin{bmatrix}
    \cos \theta &  -\sin \theta & 0 \\[3pt]
    \sin \theta & \cos \theta & 0\\[3pt]
    0 & 0 & 1\\
  \end{bmatrix}$
\end{center}

Поворот навколо декількох осей одночасно можна виразити як добуток окремих поворотів навколо цих осей:

\begin{center}
$R = R_x(\gamma) * R_y(\beta) * R_z(\alpha)$
\end{center}

\subsubsection{Матриця масштабування}

Масштабуванням є лінійне перетворення, що розтягує чи стискає об'єкти на деякі фактори масштабування вздовж осей. Воно може бути представлене у вигляді матриці перетворення наступним чином:

\begin{center}
  $S_v =
  \begin{bmatrix}
    v_x & 0 & 0  \\
    0 & v_y & 0  \\
    0 & 0 & v_z  \\
  \end{bmatrix}$
\end{center}

Нижченаведений приклад ілюструє дію матриці масштабування:

\begin{center}
  $S_vp =
  \begin{bmatrix}
    v_x & 0 & 0  \\
    0 & v_y & 0  \\
    0 & 0 & v_z  \\
  \end{bmatrix}
  \begin{bmatrix}
    p_x \\ p_y \\ p_z
  \end{bmatrix}
  =
  \begin{bmatrix}
    v_xp_x \\ v_yp_y \\ v_zp_z
  \end{bmatrix}$
\end{center}

\subsection{Матриці перетворень у графіці}

\subsubsection{Матриця моделі}

Для переходу від координат моделі до світових координат використовується матриця моделі. Зазвичай визначається як добуток матриць масштабування, повороту (на орієнтацію моделі), та перенесення (на глобальну координату моделі):

\begin{center}
  $M = T * R * S$
\end{center}

\subsubsection{Матриця виду}

Для переходу від світових координат до координат камери використовується матриця виду V. Вона залежить від координат ока $\vec e$, координат точки $\vec c$, у яку спрямована, вектору напряму вгору $\vec u$, і у звичному випадку визначається наступним чином:

\begin{equation}
  \begin{split}
    \vec{F} = \vec{c} - \vec{e}
  \end{split}
  \quad\quad
  \begin{split}
    \vec{f} = \frac{\vec F}{\|\vec{F}\|}
  \end{split}
  \quad\quad
  \begin{split}
    \vec{u'} = \frac{\vec u}{\|\vec{u}\|}
  \end{split}
\end{equation}
\begin{equation}
  \begin{split}
    \vec{s} = \vec{f} \times \vec{u'}
  \end{split}
  \quad\quad
  \begin{split}
    \vec{d} = \frac{\vec s}{\|\vec{s}\|} \times \vec{f}
  \end{split}
\end{equation}
\begin{equation}
  V = T_{-\vec{e}} * \begin{bmatrix}
    \vec{s}_0 & \vec{s}_1 & \vec{s}_2 & 0 \\
    \vec{d}_0 & \vec{d}_1 & \vec{d}_2 & 0 \\
    -\vec{f}_0 & -\vec{f}_1 & -\vec{f}_2 & 0 \\
    0 & 0 & 0 & 1 \\
  \end{bmatrix}
\end{equation}

\subsubsection{Матриця проекції}

Для переходу від координат камери до координат екрану (зокрема, для врахування відстані до камери і створення ефекту перспективи) використовується матриця проекції. У випадку перспективної проекції вона залежить від куту огляду в напрямку $y$ у градусах $\mathit{fov_y}$, відношення сторін екрану $\mathit{aspect}$, відстані від глядачу до ближньої відсікаючої поверхні $\mathit{z_{near}}$, відстані від глядачу до дальної відсікаючої поверхні $z_{far}$, і визначається наступним чином:

\begin{center}
  $f = \operatorname{cot}(\mathit{fov_y} / 2)$ \\~\\
  $P = \begin{bmatrix}
    \frac{f}{\mathit{aspect}} & 0 & 0 & 0 \\
    0 & f & 0 & 0 \\
    0 & 0 & \frac{\mathit{z_{far}}+\mathit{z_{near}}}{\mathit{z_{near}}-\mathit{z_{far}}}
      & \frac{2*\mathit{z_{far}}*\mathit{z_{near}}}{\mathit{z_{near}}-\mathit{z_{far}}} \\
    0 & 0 & -1 & 0 \\
  \end{bmatrix}$
\end{center}

\subsubsection{MVP-матриця}

Нарешті, матриця, що здійюснює весь ланцюжок перетворень від координат об'єкту до координат екрану називається $\textit{Model-View-Projection}$-матрицею і визначається як добуток матриць моделі, виду, і проекції:

\begin{center}
  $\mathit{MVP} = M*V*P$
\end{center}

Зручність представлення усіх цих перетворень однією матрицею полягає у тому, що до вертексного шейдеру, який буде обчислювати остаточні координати вершин на GPU, тепер потрібно передавати лише один параметр замість трьох.

\subsection{Кватерніони}

\subsubsection{Визначення}

Кватерніон  — гіперкомплексне число, яке реалізується в 4-вимірному просторі. Вперше описане В. Р. Гамільтоном у 1843 році.

Кватерніони використовуються як у теоретичній, так і у прикладній математиці, зокрема для розрахунку поворотів у просторі у тривимірній графіці та машинному зорі.

Кватерніон має вигляд $a + bi + cj + dk$, де $a, b, c, d$ — дійсні числа;
$i, j, k$ — уявні одиниці, що задовольняють співвідношенням
$i^2 = j^2 = k^2 = ijk = -1$, з яких випливають також наступні співвідношення: \\
\begin{center}
  $\begin{matrix}
    ij & = & -ji & = & k, \\
    jk & = & -kj & = & i, \\
    ki & = & -ik & = & j.
  \end{matrix}$
\end{center}

Часто замість $i, j, k$ використовують позначення для уявних одиниць відповідно $i_1, i_2, i_3$, а також покладають $i_0 := 1$.

Ще один, зрідка вживаний, варіант позначень: $e_0, e_1, e_2, e_3$.

Кватерніони також можна визначити через комплексні числа, використовуючи процедуру подвоєння Келі-Діксона.

\subsubsection{Сферична лінійна інтерполяція}

В комп'ютерній графіці, SLERP (spherical linear interpolation) — лінійна інтерполяція на сфері, що використовується для анімації обертання з постійною кутовою швидкістю за допомогою кватерніонів.

SLERP має геометричну інтерпретацію незалежну від кватерніонів та розмірності простору. Вона базується на тому, що довільна точка на кривій повинна представлятись у вигляді лінійної комбінації кінців кривої. Якщо простором, в якому беруться точки, буде сфера, то геодезичний відрізок, який їх з'єднує буде не евклідовим відрізком (він не належить сфері), а буде дугою великого кола на сфері.

Якщо p0 та p1 початок і кінець дуги, а t параметр, $0 \le t \le 1$.

Обчислимо $\Omega$ — кут дуги, отримаємо $\cos{\Omega} = p_0 \cdot p_1$, n-вимірний скалярний добуток одиничних векторів. Отримаємо формулу:
\begin{center}
  $\operatorname{Slerp}(p_0, \, p_1, \, t) = \frac{\sin {(1-t)\Omega}}{\sin \Omega} p_0 + \frac{\sin t\Omega}{\sin \Omega} p_1.$
\end{center}

Вона симетрична відносно кінців дуги
$\operatorname{Slerp}(p_1,p_1,t) = \operatorname{Slerp}(p_1,p_0,1-t)$.

Записавши одиничний кватерніон у вигляді $q = \cos{\Omega} + v\sin{\Omega}$, де v - тривимірний одиничний вектор, отримаємо $qt = \cos{t\Omega} + v\sin{t\Omega}$.

Записавши $q = q_1q_0^{-1}$, отримаємо

$\operatorname{Slerp}(q_0, q_1, t) = q_0^{1-t} q_1^t
=q_0 (q_0^{-1} q_1)^t
=q_1 (q_1^{-1} q_0)^{1-t}
=(q_0 q_1^{-1})^{1-t} q_1
=(q_1 q_0^{-1})^t q_0
$