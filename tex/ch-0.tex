\section*{ВСТУП}
\addcontentsline{toc}{section}{ВСТУП}

Об'єкт моделювання представляє з себе просторову варіацію вправи для розвитку короткочасної пам'яті N-Back, реалізовану у формі інтерактивної гри. Оригінальну вправу винайшов Уейн Кірхнер у 1958 році\cite{kirchner} як задачу неперервного виконання для оцінки короткочасної пам'яті з когнітивно-нейронаукових позицій.

Задача у найбільш загальній формі — назвемо її K-N-Back — полягає у тому, що суб'єкту надається ряд із K послідовностей ознак, і він повинен реагувати на повтори ознак у рамках своїх рядів, тобто, на кожному N-му кроці відповідати на питання, чи зустрічалася дана ознака для даного ряду N кроків назад. Таким чином, у термінології комп'ютерних наук, вправа полягає у тому, щоб тримати у пам'яті K буферів довжини N, на кожному кроці класти у кожний з них по новому елементу, здвигаючи усі інші і відкидуючи останні.

Найбільш популярну варіацію вправи — 2-N-Back — запропонувала Сюзанна Йеггі у 2003 році\cite{jaeggi}, як застосування до вправи парадигми подвійної задачі. Ця парадигма полягає у виконанні двох задач одночасно і подальших порівнянь із продуктивністю на кожній з задач окремо. Якщо результат погіршується, то задачі перетинаються, і вважається, що вони потребують ресурси мозку з одного і того ж класу щодо потреб з обробки інформації.

Більшість із поширених комп'ютерних реалізацій вправи моделюють її 2-N-Back варіант, інші - дозволяють у деяких діпазонах змінювати K та N, деякі підтримують більш екзотичні модифікації, та усі є двовимірними щодо найбільш популярної ознаки — позиції на дошці. У даній роботі дошка є трьовимірною і має структуру кубіка Рубіка, тобто великого куба, що складається з менших. Кожна грань цього куба спіставима із двовимірною дошкою для K-N-Back, і відкривається цілий простір до нових модифікацій, заснованих на можливості зіставлення трьовимірним рухам дошки інформації, яку користувач має прийняти до уваги.

У науковців досі виникають питання щодо конструктної валідності вправи. У той час як задача має сильну довірчу валідність і знайшла широке застосування у якості міри короткочасної пам'яті у кліничній та експериментальній практиці, існує декілька досліджень, демонструючих помірну конвергентну валідність із іншими мірами короткочасної пам'яті\cite{kane-conway}\cite{jaeggi-buschkuehl}.

Щодо такої кореляції існують дві основних гіпотези: одна полягає у тому, що вправа активізує відмінні від інших субкомпоненти короткочасної пам'яті; інша, більш критична, стверджує, що успішність виконання вправи більше залежить від звички та когнітивних процесів впізнавання. Як би там не було, науковці підкреслюють необхідність подальшого дослідження конструктної валідності N-Back.

Нейробіологічні дослідження показали\cite{neuro}, що під час виконання вправи N-Back найбільш активними є такі зони мозку: бічна премоторна кора (lateral premotor cortex), дорсальна поясна кора (dorsal cingulate cortex),  середня премоторна кора (medial premotor cortex), дорсолатеральна та вентролатеральна префронтальна кора (dorsolateral and ventrolateral prefrontal cortex), фронтальні полюси (frontal poles), медіальна і латеральна задня тім'яна кора (medial and lateral posterior parietal cortex).

\begin{comment}
У багатьох реалізаціях N-Back (наприклад, Brain Workshop) наявна конфігуруємість, у якій доступні ті чи інші модифікації класичної вправи. Якщо їх назвати мікромодифікаціями, то основну модифікацію даного продукту — просторовість — по відношенню до них можна буде назвати макромодифікацію у тому сенсі, що вона відкриває цілий вимір для нових мікромодифікацій.

Класичну вправу та деякий клас її мікромодифікацій можна узагальнити до K-N-Back, сутність якого полягає у тому, що користувачу послідовно показується ряд груп образів по K за раз, і на кожному кроці користувач повинен визначити, які з поточно-показаних K образів були показані рівно N кроків назад, натиснувши відповідні кожному з них кнопки.

У найбільш поширеному варіанті вправи — Dual-N-Back — як можна здогадатися, K дорівнює 2. Як правило, образами у реалізації такого варіанту є візуальний — підсвічена клітина на дошці, та аудіо — озвучена літера.

У даному варіанті дошка є трьовимірною і має структуру кубіка Рубіка, тобто великого куба, що складається з менших. Кожна грань цього куба є дошкою для K-N-Back, а нові мікромодифікації полягають у можливості зіставлення трьовимірним рухам дошки інформації, яку користувач має прийняти до уваги.
\end{comment}

Цільовою аудиторією є будь-які люди, які хочуть покращити короткочасну пам'ять за тих чи інших причин. У клінічній практиці класичний варіант вправи використовується для підвищення здатності до концентрації і короткочасного запам'ятовування у людей з синдромом дефіциту уваги і гіперактивності та для реабілітації здібностей у людей з травмами мозку.