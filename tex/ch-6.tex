\section{РОЗДІЛ З ОХОРОНИ ПРАЦІ}

Метою цього розділу є врахування факторів, при яких виникає небезпека ураження організму в процесі роботи з даним програмним комплексом. Даний програмний комплекс буде встановлено в офісі організації ``Abelian Solutions''. Офіс знаходиться у багатоповерховому офісному центрі та має сучасне оснащення. Колектив складається з двох працівників, що працюють за нормованим графіком.

\subsection{Характеристика організації виробництва, техніки, технології виробництва з точки зору охорони праці}

Приміщення, що буде розглядатися, знаходиться на другому поверсі чотирьохповерхового будинку. Два вікна (1,5 х 2,0) кімнати орієнтовані на схід. Як основні характеристики приміщення приймаються його геометричні розміри (площа, обсяг) і кількість працюючих у ньому людей.
Розміри аналізованого приміщення: 5 м -- довжина, 5 м -- ширина та 3.5 м -- висота.

\small\begin{longtable}{| C{4cm} | C{4cm} | C{4cm} | C{4cm} |}
  \caption{Площа та обсяг приміщення, на одного працюючого}
  \label{tab:office} \\
  \hline
  Геометрична хар-ка & Одиниця виміру & Нормативне & Фактичне \\
  \hline
  Площа, $S$ & $\textit{м}^2$ & не менше $6.0$ & $12.5$ \\
  \hline
  Обсяг, $V$ & $\textit{м}^3$ & не менше $19.5$ & $43.75$ \\
  \hline
\end{longtable}\normalsize

За даними, наведеними у таблиці \ref{tab:office}, можна зробити висновок, що геометричні розміри приміщення відповідають нормативним вимогам ДСанПіН 3.3.2.007-98.

\begin{figure}[here]
  \centering\includegraphics[scale=0.8]{./diagrams/office.png}
  \caption{План офісного приміщення\\
    1, 2, 3, 4 -- світильники; 5, 6 -- шафи\\
    7, 8 -- столи з ПЕОМ; 9, 10 -- стільці\\
    11, 12 -- вікна}
\end{figure}

\newpage

\subsection{Аналіз потенційних небезпек та розробка заходів по покращенню умов праці}

\subsubsection{Мікрокліматичні умови}

Повітряне середовище в приміщенні характеризується мікрокліматом, запиленістю повітря та його загазованістю.

Згідно ДСН 3.3.6.042-99, оптимальні значення температури, відносної вологості і швидкості руху повітря встановлюються для робочої зони виробничих приміщень з урахуванням ваги виконуваної роботи і пори року.

Температура повітря в приміщенні визначається температурою зовнішнього повітря і тепловою енергією, що виділяється всередині приміщення. Джерелами теплоти в даному приміщенні є люди, електроустаткування, а також освітлювальні прилади в темний час доби. Зовнішнім джерелом надлишкового тепла є сонячна радіація у світлий час доби. Робота, виконувана в даному приміщенні, відноситься до категорії Іа. Людиною в цьому випадку виділяється до 120 ккал теплової енергії в годину. Вологість повітря в приміщенні визначається вологістю атмосферного і видихуваного людьми повітря, а також випарами з поверхні шкіри.

У таблиці \ref{tab:climate} приведені оптимальні значення параметрів мікроклімату для категорії ваги робіт Iа, а також фактичні значення цих параметрів у розглянутому приміщенні. У приміщеннях з використанням обчислювальної техніки рекомендується застосування тільки оптимальних значень показників мікроклімату, тобто таких, при яких людина почуває себе комфортно.

\small\begin{longtable}{| C{2cm} | C{2cm} | C{2cm} | C{2cm} | C{2cm} | C{2cm} | C{2cm} |}
\caption{Оптимальні і фактичні значення параметрів мікроклімату}
\label{tab:climate} \\
\hline
& \multicolumn{3}{|c|}{Оптимальні для Iа} & \multicolumn{3}{|c|}{Фактичні} \\
\hline
Пора року & Темп., $^{\circ}C$ & Вологість, $\%$ & Швидкість повітря, $\frac{\textit м}{\textit с}$ & Темп., $^{\circ}C$ & Вологість, $\%$ & Швидкість повітря, $\frac{\textit м}{\textit с}$ \\
\hline
Тепла & 23-28 & 50-70 & 0.1 & 25-30 & 40-50 & 0.15 \\
\hline
Холодна & 22-24 & 40-60 & 0.1 & 19-22 & 40-50 & 0.1 \\
\hline
\end{longtable}\normalsize

Показники мікроклімату в приміщенні, загалом, відповідають установленим нормам, крім температури в теплий період року. Для компенсації цих надлишків можна застосувати будь-який сучасний кондиціонер з потужністю по холоду до 7.0 кВт.

Джерелами запиленості повітря в приміщенні є одяг людей і пил, що проникає з вулиці. З метою боротьби з пилом робляться регулярні вологі прибирання і провітрювання. У приміщенні немає виділення шкідливих газів. Тому що в ньому не проводиться монтажних робіт, пайки чи інших робіт, при яких виділяються шкідливі гази. Для нормалізації параметрів повітряного середовища також періодично здійснюється провітрювання приміщення і вологе прибирання. У всьому будинку діє встановлена загально обмінна витяжна вентиляція.

\subsubsection{Вимоги до освітленості}

Природне освітлення в розглянутому приміщенні представлено системою однобічного бічного освітлення з двома вікнами з загальної  шириною 3.0 м. Висота вікна складає 2 м. Загальна площа вікон – 6 $\text{м}^2$.

Штучне освітлення приміщення здійснюється за допомогою системи загального освітлення. Зорові умови праці при штучному освітленні характеризуються значенням освітленості, показником чи засліпленості дискомфорту і коефіцієнтом пульсації освітленості. При роботі з відеотерміналами необхідна, також, оцінка значення яскравісного контрасту.

У розглянутому приміщенні, використовується система загального рівномірного висвітлення. У приміщенні мається два стельових світильників типу Л201-03, у кожному з яких знаходиться по чотири люмінсцентні лампи ЛБ-40 потужністю 40 Вт (світловий потік 3200 лм) кожна.
Фактична освітленість дорівнює 398 лк. Порівнявши це значення з нормою освітленості згідно ДСанПіН 3.3.2.007-98 (200-400 лк), маємо, що це задовольняє заданим вимогам.

\subsubsection{Вимоги до рівнів шуму та вібрації}

Відповідно до ДСН 3.3.6.037-99 рівень шуму в приміщенні для працюючих за ПЕОМ не повинний перевищувати 50 дБА для режиму налагодження і 60 дБА для режиму введення інформації.
Приміщення розташоване вікнами у двір і знаходиться далеко від проїжджої частини вулиці. Основними джерелами шуму в приміщенні є устаткування і люди. Розглянуте офісне приміщення не призначена для прийому відвідувачів і тому в ній не спостерігається великого скупчення людей. Тому основним джерелом шуму є комп'ютерна техніка.

Рівень шуму в приміщенні складає 46.3 дБА, що не перевищує норму.

\subsubsection{Випромінювання при роботі з обчислювальною технікою}

До джерел електромагнітного поля будь-якої інтенсивності відносяться технічні засоби та пристрої, що споживають, перетворюють, або виробляють електромагнітну енергію. Гранично допустимими нормами випромінювання є $\text{Е}_{\text{пду}} = 25 \frac{\text В}{\text м}$ та $\text{В}_{\text{пду}} = 250 \text{нТл}$ при частоті випромінювання 5 -- 2000 Гц, $\text{E}_{\text{пду}} = 2.5 \frac{\text B}{\text м}$ та $\text{B}_{\text{пду}} = 25 \text{нТл}$ при 2 -- 400 кГц а також $\text{E}_{\text{пду}} = 2.5 \frac{\text B}{\text м}$ та $\text{B}_{\text{пду}} = 25 \text{нТл}$ при 400 кГц – 500 МГц.

Наявність у приміщенні декількох комп'ютерів з допоміжною апаратурою і системою електроживлення створює складну картину електромагнітного поля. Згідно з характеристиками апаратури, маємо наступні дані (таблиця \ref{tab:electromagnetic}).

\small\begin{longtable}{| C{5cm} | C{6cm} | C{5cm} |}
  \caption{Значення електромагнітних полів на робочих місцях}
  \label{tab:electromagnetic} \\
  \hline
  Найменування вимірюваних параметрів & Діапазон частот 5 Гц - 2 кГц & Діапазон частот 2 - 400 кГц \\
  \hline
  $\text{Е}_{\text{пду}}, (\frac{\text В}{\text м})$ & 16.0-18.0 & 1.68-2.00 \\
  \hline
  $\text{B}_{\text{пду}}, (\text{нТл})$ & 80.0-100.0 & 10.6-12.00 \\
  \hline
\end{longtable}\normalsize

Як бачимо, отримані дані не перевищують гранично допустимі дані для електромагнітного випромінювання.

\subsubsection{Небезпека враження людини електричним струмом}

Потенційну небезпеку для людини представляють електричні прилади й установки, що живляться небезпечною для життя людини напругою 220 В. Ураження людини електричним струмом може відбутися в результаті дотику до відкритих струмоведучих частин при ушкодженні ізоляції мережних шнурів, при пробої при короткому чи замиканні в результаті неправильних дій самої людини.

Дана кімната по ступені небезпеки поразки електричним струмом відноситься до приміщень без підвищеної небезпеки. Споживачами електроенергії є ПЕОМ, дисплей, джерела висвітлення. Сучасні ПЕОМ розробляються відповідно до вимоги по електробезпеки для побутових приладів, тому які-небудь додаткові засоби захисту від поразки електричним струмом не вимагаються. У приміщенні застосовуються наступні засоби захисту: недоступність струмоведучих джерел;  малі напруги; захисне відключення у блоках живлення; ізоляція струмоведучих частин; попереджувальні написи та індикатори; занулення струмопровідних частин.

Таким чином, норми електробезпеки згідно ПУЕ-2009 виконані.

\subsubsection{Пожежна безпека}

У досліджуваному приміщенні є в наявності тільки тверді і волокнисті пальні речовини: дерево, папір, тканина. Таким чином, робочі зони приміщення відноситься до класу П-IIа, а приміщення до категорії  по пожежонебезпеці В згідно з НАПБ Б.03.002-2007. Можливими причинами пожежі в приміщенні є несправність електроустаткування і порушення протипожежного режиму (використання побутових нагрівальних приладів, паління). Для гасіння пожежі в коридорі розташовані вуглекислотні вогнегасники ОУ–5, а в кожній кімнаті, де встановлені комп'ютери, додатково знаходяться вуглекислотні вогнегасники ОУ–5 (2 шт). Також на сходовій клітині розташований пожежний кран. Така кількість, розташування та умови зберігання первинних засобів пожежогасіння відповідають вимогам НПАОП 0.00-1.28-10. Будинок має два еваковиходи: через головний хід і спеціальний еваковихід. Шляху евакуації відповідають установленим нормам ДНАОП 0.01-1.01-95. Для попередження пожежі в приміщеннях, згідно вимогам ДБН  передбачений пристрій системи пожежної сигналізації. Теплові і димові повідомники встановлюються на стелях відповідних приміщень.

 Таким чином усі фактори пожежної безпеки задовольняють вимогам встановлених норм, згідно ``Правил пожежної безпеки в Україні''.

\subsection{Інструкція з охорони праці}

\subsubsection{Вимоги безпеки перед початком роботи}

Перед початком роботи слід дотримуватись таких вимог:
\begin{itemize}
  \item увімкнути систему кондиціювання повітря в приміщенні;
  \item впевнитись, що на робочому місці відсутні сторонні предмети, і що все обладнання і блоки ПЕОМ з'єднані з системним блоком;
  \item перевірити загальний стан апаратури, перевірити справність електропроводки, з'єднувальних шнурів, штепсельних вилок, розеток, заземлення захисного екрана;
  \item відрегулювати освітленість робочого місця, яскравість свічення екрана ВДТ, мінімальний розмір світної точки, фокусування, контрастність;
  \item у разі виявлення будь-яких несправностей роботу не розпочинати, повідомити про це керівника робіт.
\end{itemize}

\subsubsection{Вимоги безпеки під час роботи з відеодисплейним терміналом}

Під час роботи з відеодисплейним терміналом слід дотримуватись таких вимог:
\begin{itemize}
  \item необхідно стійко розташувати клавіатуру на робочому столі, не допускаючи її хитання.	Має бути передбачена можливість її поворотів та переміщень;
  \item забороняється самостійно ремонтувати апаратуру, в якому кінескоп знаходиться під високою напругою (близько 25 кВ) та класти будь-які предмети на апаратуру комп'ютера, напої на клавіатуру або поруч з нею;
  \item для зняття статичної електрики рекомендується час від часу, доторкатися до металевих поверхонь (батарея ЦО тощо);
  \item перерви в роботи по 20 хвилин доцільно встановити через 2 години від початку зміни, через 1,5 та 2,5 години після обідньої перерви.
\end{itemize}

\subsubsection{Вимоги безпеки після закінчення роботи}

Після закінчення роботи слід дотримуватись таких вимог:
\begin{itemize}
  \item закінчити роботу та зберегти дані;
  \item вимкнути принтер, інші периферійні пристрої та ПЄВМ;
  \item прибрати робоче місце та ретельно вимити руки теплою водою з милом;
  \item вимкнути кондиціонер, освітлення і загальне електроживлення офісу.
\end{itemize}

%%%%%%%%%%%%%%%%%%%%%%%%%%%%%%%%%%%%%%%%%%%%%%%%%%%%%5

\begin{comment}
\subsection{Загальні положення}
\begin{itemize}
  \item На місці тренування користувач отримує первинний інструктаж з охорони праці
    та проходить навчання вправі і правилам експлуатації використовуваного устаткування.
  \item Користувач зобов'язаний дотримуватися правил внутрішнього розпорядку,
    режиму тренувань і відпочинку і строго дотримуватися інструкції з охорони праці
    при роботі на комп’ютері.
  \item Користувач зобов'язаний знати та дотримуватися правил особистої гігієни:
    \begin{itemize}
      \item Приходити на роботу в чистому одязі і взутті.
      \item Постійно стежити за чистотою тіла, рук, волосся.
      \item Мити руки з милом після відвідування туалету,
        дотику забруднених предметів, після закінчення тренувальння.
    \end{itemize}
  \item Забороняється зберігати на місці тренування
    пожежо- та вибухонебезпечні речовини.
\end{itemize}
\newpage
\subsection{Вимоги безпеки перед початком роботи}
\begin{itemize}
  \item Перевірити оснащеність робочого місця, справність обладнання,
    електропроводки на видимі пошкодження. При несправності
    повідомити інструкторові.
  \item Перевірити зовнішнім оглядом достатність освітлення
    і справність вимикачів і розеток.
  \item Звільнити робоче місце від сторонніх преметів.
\end{itemize}
\newpage
\subsection{Вимоги безпеки під час роботи}
\begin{itemize}
  \item Виконувати тільки ті дії, за якими пройдено інструктаж з охорони праці.
  \item Не доручати управління тренажером під час сеансу стороннім особам.
  \item Під час знаходження на місці тренування користувачі не повинні здійснювати дії,
    які можуть спричинити за собою нещасні випадки:
    \begin{itemize}
      \item Не гойдатися на стільці.
      \item Не торкатися оголених проводів.
      \item Не працювати на обладнанні мокрими руками.
      \item Не розмахувати гострими і ріжучими предметами.
    \end{itemize}
  \item Дотримуватися правил переміщення в приміщенні і на території відділу освіти,
    користуватися тільки встановленими проходами. Не захаращувати встановлені проходи.
  \item Внаслідок того, що весь процес тренування за комп'ютером, необхідно кожні дві години
    відволікатися і робити перерву 15 хвилин, для зниження стомлюваності
    загальнофізичного характеру.
  \item Користувач повинен повідомляти свого інструктора про будь-які ситуації,
    загрозливі життю і здоров'ю людей, про кожний нещасний випадок,
    про погіршення стану свого здоров'я,
    у тому числі про прояв ознак гострого захворювання.
\end{itemize}
\newpage
\subsection{Вимоги безпеки після закінчення роботи}
\begin{itemize}
  \item Провести прибирання робочого місця.
  \item Перевірити протипожежний стан кабінету.
  \item Закрити вікна, світло, вимкнути кондиціонер, закрити двері.
\end{itemize}
\newpage
\subsection{Вимоги безпеки в аварійних ситуаціях}
\begin{itemize}
  \item В аварійній обстановці слід сповістити про небезпеку оточуючих людей
    і діяти відповідно до плану ліквідації аварій.
  \item У разі виникнення спалаху або пожежі, необхідно негайно повідомити
    про це в пожежну частину, окриком попередити оточуючих людей і вжити
    заходів для гасіння пожежі.
  \item У випадках травмування і несправностей в устаткуванні працівник негайно
    припиняє роботу і повідомляє своєму безпосередньому начальникові про те,
    що трапилося, надає собі або іншому працівнику першу долікарську допомогу,
    за необхідності викликає швидку допомогу.
  \item У ситуаціях, які загрожують життю та здоров'ю - покинути небезпечну ділянку.
\end{itemize}
\end{comment}

\textbf{Висновок до розділу.} У даному розділі проект було розглянуто з точки зору охорони праці -- були проаналізовані потенційні небезпеки та розроблені заходи з покращення умов праці: сформульовано вимоги до мікроклімату, освітленості, рівнів шуму та вібрації, розглянуті небезпеки випромінювання, враження електричним струмом, та пожежі. Була описана інструкція з охорони праці.