\section{Розділ з охорони праці}
\subsection{Загальні положення}
\begin{itemize}
  \item На місці тренування користувач отримує первинний інструктаж з охорони праці
    та проходить навчання вправі і правилам експлуатації використовуваного устаткування.
  \item Користувач зобов'язаний дотримуватися правил внутрішнього розпорядку,
    режиму тренувань і відпочинку і строго дотримуватися інструкції з охорони праці
    при роботі на комп’ютері.
  \item Користувач зобов'язаний знати та дотримуватися правил особистої гігієни:
    \begin{itemize}
      \item Приходити на роботу в чистому одязі і взутті.
      \item Постійно стежити за чистотою тіла, рук, волосся.
      \item Мити руки з милом після відвідування туалету,
        дотику забруднених предметів, після закінчення тренувальння.
    \end{itemize}
  \item Забороняється зберігати на місці тренування
    пожежо- та вибухонебезпечні речовини.
\end{itemize}
\newpage
\subsection{Вимоги безпеки перед початком роботи}
\begin{itemize}
  \item Перевірити оснащеність робочого місця, справність обладнання,
    електропроводки на видимі пошкодження. При несправності
    повідомити інструкторові.
  \item Перевірити зовнішнім оглядом достатність освітлення
    і справність вимикачів і розеток.
  \item Звільнити робоче місце від сторонніх преметів.
\end{itemize}
\newpage
\subsection{Вимоги безпеки під час роботи}
\begin{itemize}
  \item Виконувати тільки ті дії, за якими пройдено інструктаж з охорони праці.
  \item Не доручати управління тренажером під час сеансу стороннім особам.
  \item Під час знаходження на місці тренування користувачі не повинні здійснювати дії,
    які можуть спричинити за собою нещасні випадки:
    \begin{itemize}
      \item Не гойдатися на стільці.
      \item Не торкатися оголених проводів.
      \item Не працювати на обладнанні мокрими руками.
      \item Не розмахувати гострими і ріжучими предметами.
    \end{itemize}
  \item Дотримуватися правил переміщення в приміщенні і на території відділу освіти,
    користуватися тільки встановленими проходами. Не захаращувати встановлені проходи.
  \item Внаслідок того, що весь процес тренування за комп'ютером, необхідно кожні дві години
    відволікатися і робити перерву 15 хвилин, для зниження стомлюваності
    загальнофізичного характеру.
  \item Користувач повинен повідомляти свого інструктора про будь-які ситуації,
    загрозливі життю і здоров'ю людей, про кожний нещасний випадок,
    про погіршення стану свого здоров'я,
    у тому числі про прояв ознак гострого захворювання.
\end{itemize}
\newpage
\subsection{Вимоги безпеки після закінчення роботи}
\begin{itemize}
  \item Провести прибирання робочого місця.
  \item Перевірити протипожежний стан кабінету.
  \item Закрити вікна, світло, вимкнути кондиціонер, закрити двері.
\end{itemize}
\newpage
\subsection{Вимоги безпеки в аварійних ситуаціях}
\begin{itemize}
  \item В аварійній обстановці слід сповістити про небезпеку оточуючих людей
    і діяти відповідно до плану ліквідації аварій.
  \item У разі виникнення спалаху або пожежі, необхідно негайно повідомити
    про це в пожежну частину, окриком попередити оточуючих людей і вжити
    заходів для гасіння пожежі.
  \item У випадках травмування і несправностей в устаткуванні працівник негайно
    припиняє роботу і повідомляє своєму безпосередньому начальникові про те,
    що трапилося, надає собі або іншому працівнику першу долікарську допомогу,
    за необхідності викликає швидку допомогу.
  \item У ситуаціях, які загрожують життю та здоров'ю - покинути небезпечну ділянку.
\end{itemize}