\section{ІНФОРМАЦІЙНЕ ЗАБЕЗПЕЧЕННЯ}
\subsection{Вхідні дані}

Вхідні дані є сенс розділяти на вхідні дані графічного інтерфейсу і вхідні дані програми моделювання. До перших належить лише профіль, відносно якого здійснюється конфігурування і перегляд результатів, а другі наведені нижче:

\begin{itemize}
  \item режим тренування;
  \item довжина буферу ознак N;
  \item додаткові просторові перетворення;
  \item додаткові ознаки;
  \item графічні опції;
  \item таймінги.
\end{itemize}

\subsection{Вихідні дані}

Вихідні дані також розрізняються для інтерфейсу користувача та програми моделювання. Вихідними даними програми моделювання є характеристики тренувальної сесії, наведені нижче, а вихідними даними інтерфейсу користувача є список наборів характеристик тренувальних сесій для даного профілю.

Характеристики тренувальної сесії:
\begin{itemize}
  \item кількість правильних відповідей;
  \item кількість неправильних відповідей;
  \item кількість пропущених відповідей.
\end{itemize}

До правильних відповідей входять як правильні сигналізування (натисканням на відповідну кнопку) повторів ознак, так і правильна бездіяльність у випадках, коли поточно-показана ознака не є повтором. Неправильні відповіді щодо цього аспекту розділені для більшої деталізації. Очевидно, що загальну кількість випробувань можна отримати як суму цих трьох кількісних характеристик.

\subsection{Опис структури бази даних}

У системі використовується реляційна база даних SQLite такої структури (рисунок \ref{fig:db}):
\begin{itemize}
  \item у таблиці ``Profile'' зберігаються профілі у формі записів, що утримують ім'я профілів, та їх налаштування;
  \item таблиця ``ExerciseSession'' утримує записи характеристик тренувальних сесій, включаючи тривалість, профіль, під яким дана сесія була виконана, та глобальну сесію застосунку, під час якої вона була виконана;
  \item таблиця ``ApplicationSession'' зберігає глобальні сесії застосунку (період від увімкнення програми конфігурування та перегляду історії до її вимкнення) із їх тривалістю;
  \item таблиця ``Global'' використовується для утримання простих глобальних зміних, що представимі у вигляді відображення ключ-значення, де і ключ, і значення є строковими -- таких як останній активний профіль.
\end{itemize}
\newpage
\begin{figure}[here]
  \centering
  \includegraphics[scale=0.6]{./diagrams/db.eps}
  \caption{Схема бази даних}
  \label{fig:db}
\end{figure}

\textbf{Висновок до розділу.} У даному розділі було розглянуто інформаційне забезпечення системи: декомпозовані за застосунком і описані множини вхідних і вихідних даних, проаналізована, описана, і візуалізована схемою структура бази даних.