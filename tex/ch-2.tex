\section{Інформаційне забезпечення}
\subsection{Вхідні дані}

Вхідні дані є сенс розділяти на вхідні дані графічного інтерфейсу і вхідні дані програми моделювання. До перших належить лише профіль, відносно якого здійснюється конфігурування і перегляд результатів, а другі наведені нижче:

\begin{itemize}
  \item Режим тренування
  \item Довжина буферу ознак N
  \item Додаткові просторові перетворення
  \item Додаткові ознаки
  \item Графічні опції
  \item Таймінги
\end{itemize}

\subsection{Вихідні дані}

Вихідні дані також розрізняються для інтерфейсу користувача та програми моделювання. Вихідними даними програми моделювання є характеристики тренувальної сесії, наведені нижче, а вихідними даними інтерфейсу користувача є список наборів характеристик тренувальних сесій для даного профілю.

Характеристики тренувальної сесії:
\begin{itemize}
  \item Кількість правильних відповідей
  \item Кількість неправильних відповідей
  \item Кількість пропущених відповідей
\end{itemize}

До правильних відповідей входять як правильні сигналізування (натисканням на відповідну кнопку) повторів ознак, так і правильна бездіяльність у випадках, коли поточно-показана ознака не є повтором. Неправильні відповіді щодо цього аспекту розділені для більшої деталізації. Як нескладно здогадатися, загальну кількість випробувань можна отримати як суму цих трьох кількісних характеристик.

\subsection{Опис структури бази даних}

У системі використовується нереляційна база даних типу key-value store такої структури:
\begin{itemize}
  \item Список ключів тренувальних сесій
  \item Співставлення кожному ключу сесії даних тренування
  \item Співставлення кожному ключу сесії її дати
  \item Співставлення кожному ключу сесії її конфігурації
  \item Співставлення кожній конфігурації множини ключів сесій
\end{itemize}

\begin{figure}[here]
  \centering
  \includegraphics[scale=0.6]{./diagrams/db.eps}
  \caption{Структура бази даних}
\end{figure}