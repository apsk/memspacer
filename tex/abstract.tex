\renewcommand{\abstractname}{AНОТАЦІЯ}
\begin{abstract}
  \textbf{Структура та обсяг роботи.}
  Пояснювальна записка складається із 6 розділів, містить \ESKDtotal{page} сторінок, 15 рисунків, \LTno\ таблиць, 1 додаток, та \ESKDtotal{bibitem} посилань.

Дипломний проект присвячений комплексу задач розширення функціональності тренажеру короткочасної пам’яті N-Back. Розробка призначена для будь-яких людей, які хочуть покращити короткочасну пам'ять і корельовані із нею характеристики за тих чи інших причин.

У розділі загальних положень наведено опис принципів K-N-Back задач, варіації реалізації, наведено опис функціональної моделі, характеристику існуючих аналогів, що були розроблені для браузерів, мобільних платформ та PC.

У розділі інформаційного забезпечення описано вхідні та вихідні дані, наведено структуру БД типу SQL.

У розділі математичного забезпечення наведено рішення двох задач: задачі переходу від координат моделі до координат екрану та задачі анімації поворотів у просторі з постійною кутовою швидкістю. Задля вирішення першої задачі було використано Model-View-Projection матриці, а друга задача була вирішена за допомогою сферичної лінійної інтерполяції кватерніонів.

У розділі програмне та технічне забезпечення описано засоби, обрані для розробки програм моделювання і конфігурування. Наведений опис архітектури програмного забезпечення.

У технологічному розділі наведено керівництво користувача по роботі з системою через інтерфейс користувача та виконуваний файл тренажеру. Наведено схему виконання випробувань.

Розділ охорони праці включає опис робочого місця програміста що здійснював розробку системи, наведено опис шкідливих та небезпечних факторів, що можуть виникати на робочому місці. Також наведено оптимальні характеристики та параметри робочого місця, техніка безпеки роботи з обладнанням.

ЗАДАЧА N-BACK, КОРОТКОЧАСНА ПАМ'ЯТЬ, КОНЦЕНТРАЦІЯ УВАГИ, ІНТЕРАКТИВНА ВПРАВА
\end{abstract}

\newpage

\renewcommand{\abstractname}{ABSTRACT}
\begin{abstract}
  \textbf{Project structure and scope.}
  Paper contains 6 sections, \ESKDtotal{page} pages, 15 figures, \LTno\ tables, \ESKDtotal{appendix} appendix, and \ESKDtotal{bibitem} references.

The diploma thesis is dedicated to N-Back exercise functionality extension tasks. The resulting system is intended to be used by any people who want to improve their short-term memory and correlated characteristics for whatever reasons.

In the generalities chapter there are descriptions of principles of K-N-Back tasks, implementation variants, the system's functional model, characteristics of analogues which have been developed for browsers, mobile platforms, and PC's.

In the informational support chapter there are descriptions of system's inputs, outputs, and database structure.

In the math chapter there are solutions of two tasks: the task of transforming coordinates from model space to screen space, and the task of animating spatial rotations with fixed angular velocity. The first task was solved with the Model-View-Projection matrices, and the second -- with spherical linear interpolation of quaternions.

In the software and technical support chapter there are overview of development tools used for creation of system's modeling and configuration applications. System's architecture is described.

In the technological chapter there is user manual covering system usage both via user interface application and modeling application's binary. System's testing scheme is given.

In the occupational safety chapter there are descriptions of system developer's workplace and of dangerous and harmful conditions which can take place. Also there are descriptions of optimal workplace conditions and instructions of safe technology use.

N-BACK TASK, SHORT-TERM MEMORY, CONCENTRATION, INTERACTIVE EXERCISE
\end{abstract}